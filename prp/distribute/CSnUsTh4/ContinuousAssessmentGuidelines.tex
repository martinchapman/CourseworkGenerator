%&pdflatex 
\documentclass[11pt]{article}
\usepackage{geometry}                % See geometry.pdf to learn the layout options. There are lots.
\usepackage{hyperref}
\geometry{letterpaper}                   % ... or a4paper or a5paper or ... 

\setlength{\parindent}{0pt}
\setlength{\parskip}{\baselineskip}%

\title{Programming Practice (PRP): \\ Continuous Assessment Guidelines}
%\author{Martin Chapman (martin.chapman@kcl.ac.uk)}
\date{}                                           % Activate to display a given date or no date

\begin{document}
\maketitle
%\section{}
%\subsection{}

Please read the following continuous assessment guidelines carefully:

\begin{itemize}

\item 15\% of your mark for PRP is based upon continuous assessment of your progress throughout the semester.

\item Assessment is in the form of four coursework tasks, released periodically throughout the semester. These tasks, collectively, contribute to the full 15\% of your grade, although not all are weighted equally. Thus, if you do not submit all four individual pieces of assessed work you significantly limit your ability to pass PRP.

\item Each coursework task represents the very \emph{minimum} understanding you should have of a topic in PRP. Always aim to understand as much about a topic as possible, in order to be prepared for the written examination.

\item A \emph{week}, for PRP continuous assessment, starts and ends on a \emph{Saturday} evening, at five minutes to midnight. Weeks in PRP are either \emph{coursework release} weeks, \emph{coursework assessment} weeks or \emph{normal lab} weeks. A \emph{coursework assessment} week \emph{directly follows} a \emph{coursework release} week. So, for example, 

	\begin{itemize}
		\item At 23.55 on Saturday 1st, a \emph{coursework release} week would begin; 
		\item At 23.55 on Saturday 8th that release week would end a \emph{coursework assessment} week would begin. 
	\end{itemize}

During a \emph{normal lab} week, there will be no assessment related activities. 

\item Each coursework task comprises a series of exercises. All coursework answers \emph{must} be your own work, although you are not restricted from discussing potential solutions with your peers, asking your lab demonstrator, consulting online resources for ideas, or the PRP course material. 

\item At the start of a \emph{coursework release} week, a download link will appear on KEATS, from which you can download the coursework tasks. All work must be completed before the end of the \emph{coursework release} week, and the start of the \emph{coursework assessment} week. \textbf{Important! If you cannot download your assessed work, see the troubleshooting section at the end of this document.}

\item Your lab sessions during a \emph{coursework release} week is designed to support you in the completion of your coursework tasks, although you \emph{must} start completing these tasks before your lab session to give yourself a full week to complete them. A release week ends for everyone at the same time, regardless of when your lab is scheduled.

\item During a \emph{coursework assessment} week, you will be examined on the work you completed and submitted during the release week. This examination will take place during your lab session, and will be conducted by your lab demonstrator. Your demonstrator will ask you detailed questions about your solutions, and general questions about topics related to the coursework exercises, in order to inform their marking of your code. Make sure you can competently explain how any code you have written works to your examiner, in order to maximise your marks. As your lab session is your {\em assessment opportunity} for the work, \textbf{if you do not attend, you will receive a mark of 0}, regardless of whether you submitted the work on KEATS.

\item Attending a lab sessions but not taking reasonable steps to make your presence known to your examiner will be treated as equivalent to not attending the session outright. Reasonable steps include responding when your name is called for assessment and enquiring if you do not believe your name has been called when it should have been. Being late to your assessment, or leaving before your conversation with your examiner, will also be considered equivalent to not attending. If, for whatever reason, you do not respond when your name is first called for assessment, your examiner may call your name for a second time, but they are not obligated to do this. 

\item Any code produced as a product of a coursework exercises \emph{must be submitted to KEATS} before the end of the \emph{coursework release} week. You will find the submission feature \emph{below} the coursework download link on KEATS, for that week. Read the submission instructions carefully in order to determine the file format that should be used for your submission. Work that is submitted in the wrong format will \textbf{not} be marked. If you are required to submit multiple files, please submit them as one compressed file. When compressing files it is essential that you do not use any non-standard compression formats. Instead, only compress your work as a ZIP (.zip), a RAR (.rar) or a TAR (.tar.gz) file. Work compressed in any other format will \textbf{not} be marked. If you do not submit the code required for a particular question to KEATS before the end of the \emph{coursework release} week, your demonstrator will give you a mark of 0 for that question.

\item All submitted work will be checked, in order to ensure that it meets the requirements of the question to which it pertains. If it does not, your mark will be capped for that question. If you feel that your work has been incorrectly capped, please be prepared to provide evidence to support your claim by presenting to your examiner the \emph{original work you submitted to KEATS}. If you are unable to resolve your claim with your examiner, you will be referred to one of the module leads. In addition, if you submit your work late, but within 24 hours, your assessment will be conducted by one of the module leads, and you will receive a mark of at most 40\% for that piece of work.

\item If due to illness or other good cause you miss the coursework submission deadline, or fail to attend your coursework assessment lab, then please complete a \href{http://www.kcl.ac.uk/college/policyzone/index.php?id=280}{Mitigating Circumstances Form} and submit this with evidence to Katherine Calvert in the departmental office, or via email to \url{ug-informatics@kcl.ac.uk}. You should submit the form as soon as possible, ideally before your deadline or lab. For further guidance on this, please contact your personal tutor.

\emph{If you have any questions about the structure of the continuous assessment in PRP, please email martin.chapman@kcl.ac.uk.}

\end{itemize}

\section{Troubleshooting}

When you click the `Download Coursework X' link, you should be taken to a page with a clear link saying `Download'.

Due to inevitable browser compatibility issues, this may not always occur. If you are unable to download your coursework, please try the following (in order):

\begin{enumerate}

	\item Refresh the window, or click back and re-select the download link.
	
	\item Resize the window, in order to refresh any content that may need to be dynamically loaded.

	\item If these above two methods do not work, inform a member of staff, or a lab demonstrator \emph{immediately}.

\end{enumerate}

If a `Download' link appears, but nothing happens when it is pressed, please try the following (in order):

\begin{enumerate}
	
	\item Refresh the window, or click back and re-select the download link.
	
	\item Right click on the `Download' link, and click `Save Target As'. Save the target PDF somewhere in your file system.
	
	\item If these above two methods do not work, inform a member of staff, or a lab demonstrator \emph{immediately}.
	
\end{enumerate}
 
If you experience any other unexpected behaviour, please email martin.chapman@kcl.ac.uk, immediately.
  
\end{document}
