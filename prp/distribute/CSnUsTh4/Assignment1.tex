%&pdflatex 
\documentclass[11pt]{article}

\usepackage{geometry}                % See geometry.pdf to learn the layout options. There are lots.
\geometry{letterpaper}                   % ... or a4paper or a5paper or ... 

\setlength{\parindent}{0pt}
\setlength{\parskip}{\baselineskip}%

\title{Programming Practice (PRP), Coursework Exercise 1 (13\%, 10 marks)}
%\author{Martin Chapman (martin.chapman@kcl.ac.uk)}
\date{}                                           % Activate to display a given date or no date

\begin{document}
\maketitle
%\section{}
%\subsection{}

\textbf{Please read the document marked `Continuous Assessment Guidelines' carefully, before attempting any piece of coursework.}

\emph{This assignment counts for 13\% of your mark for PRP continuous assessment, and is the first of four.}

\emph{The release week for this assignment starts 26th September, at 23:55, and ends 3rd October, at 23:55. All submissions must occur before the end of the release week.}

\emph{If you have any questions about the structure of this assessment, please email \\ martin.chapman@kcl.ac.uk.}

For this weeks assessment, consider the following tasks:

\begin{enumerate}

	\item Download the program `HelloWorld.java' from the PRP KEATS page (located below this assessment's download link), fix it, compile it, and run it (2 marks).
	
	\item Write a program that prints the words _Firstname_ _Surname_ to the screen (8 marks).

\end{enumerate}

\textbf{Once you have completed these questions, you must place all the code you have produced (including any downloaded and modified from KEATS) into a folder, name this folder `Exercise1', zip it and submit it to KEATS. Please note that you should only submit plain text files with a .java extension for assessment (so no proprietary formats such as PDF or Rich Text).}

\end{document}  
